\documentclass[12pt]{article}
\usepackage{fullpage}
\usepackage{amsthm}
\usepackage{amsfonts,amsmath, amssymb,latexsym,mathrsfs}
\usepackage[margin=1.15in]{geometry}
\usepackage{enumitem}
\setlength{\parindent}{0pt}
\usepackage{tikz-cd}
\usepackage{fancyhdr}




\theoremstyle{definition}
\newtheorem{thm}{Theorem}[section]

\newtheorem{prop}{Proposition}[section]


\theoremstyle{definition}
\newtheorem{definition}{Definition}[section]

\theoremstyle{remark}
\newtheorem*{remark}{Remark}

\theoremstyle{definition}
\newtheorem{example}{Example}[section]


\theoremstyle{definition}
\newtheorem{lem}{Lemma}[section]


\theoremstyle{definition}
\newtheorem{cor}{Corollary}[section]


\date{}
\title{Chapter 5: Definite integral}


\begin{document}
\maketitle







\begin{definition}
	A definite integral of $f$ from $a$ to $b$ is definted as 
\[\int_a^bf(x)dx= \lim_{n\rightarrow \infty} \sum_{i=1}^n f(x_i)\Delta x\text{\bf (Limit of Right-hand sum)}\]
or 
\[\int_a^bf(x)dx= \lim_{n\rightarrow \infty} \sum_{i=0}^{n-1} f(x_i)\Delta x\text{\bf (Limit of Left-hand sum)}\]

Here, Left-hand sum and Right-hand sum are equal after taking limits and it the the so-called Riemann Sum.
\end{definition}



There are two more type of Riemann sum I would like to discuss in the future, which is the Mid sum and the Trapezoidal sum. 

I will only give definition here.
\[\int_a^bf(x)dx\approx  \sum_{i=0}^{n-1} f(\frac{x_i+x_{i+1}}{2})\Delta x\text{\bf (Mid sum)}\]
and
\[\int_a^bf(x)dx\approx \sum_{i=0}^{n-1} \frac{f(x_i)+f(x_{i+1})}{2}\Delta x\text{\bf (Trapezoid sum)}\]

In all these Riemann sum we discussed, we are assuming $\Delta(x)=\dfrac{b-a}{n}$, thus as $n\to \infty$, $\Delta x \to 0$.

Note that \[\frac{LEFT(n)+RIGHT(n)}{2}=TRAP(n)\], \[MID(n)\neq TRAP(n)\].

\newpage

\begin{remark}
	Properties of definite integral:
	\begin{enumerate}
		\item \[\int^a_b f(x) dx = -\int^b_a f(x) dx\]
		\item \[\int^a_b f(x) dx+\int^b_c f(x) dx=\int^a_c f(x) dx\]
		\item \[\int^a_b (f(x)\pm g(x)) dx=\int^a_b f(x) dx \pm \int^a_b g(x) dx\]
		\item \[\int^a_b cf(x) dx = c \int^a_b f(x) dx\]
		\item Symmetry due to the oddity of the function.
	\end{enumerate}
\end{remark}

\hrulefill

\begin{remark}
	Interpretation of Define Integral as Area under graph of $f$ between $x=a$ and $x=b$, counting positivity.
\end{remark}

Important cases discussed on Friday's course:
\begin{enumerate}
	\item Compute \[\int^{1}_{-1}\sqrt{1-x^2}dx\]
	\item How about \[\int^{1}_{-1}(\sqrt{1-x^2}-1)dx?\]
	\item Maybe try\[\int^{0.5}_{-0.5}\tan(x)dx\]
\end{enumerate}

Find out the answer yourself only geometrically, even you know more techniques!



More Importantly there are two major topics I want to mention here and maybe discuss:

\begin{itemize}
	\item When is the estimation done by Riemann sum a underestimate/overestimate?
	
\textbf{It is also covered in 7.5, check it out and try problem 2.}
	
	\item Error estimation
	
	Think about the case where you know that $f(x)$ lies between any pair of $LEFT(n)$ and $RIGHT(n)$, then we see that $|LEFT(n) - f(x)|<|LEFT(n)-RIGHT(n)|=(f(b)-f(a))\Delta x$. This usually gives a bound for $n$.
\end{itemize}

\hrulefill
\begin{thm}
\textbf{The Fundamental Theorem of Calculus} is basically the theorem defined below.

If $f$ is continuous on interval $[a,b]$ and $f(t)=F'(t)$, then \[\int^b_a f(t) dt = F(b)-F(a).\]
\end{thm}
\hrulefill

\textbf{Application of Definite Integral}

\begin{enumerate}
	\item Average value of function $f(x)$ in $[a,b]$ is \[\frac{1}{b-a} \int_{a}^{b}f(x)dx\].
	
	Here, think about the integration is analogue to summation in the discrete world, then this average value is the analogue of \[x_{\text{average}}=\frac{x_1+x_2+\ldots+x_n}{1+1+1+\dots}=\frac{x_1+x_2+\ldots+x_n}{n}.\] where in the integration case is actually \[x_{\text{average}}=\frac{\lim_{n\rightarrow \infty} \sum_{i=0}^{n-1} f(x_i)\Delta x}{\lim_{n\rightarrow \infty} \sum_{i=0}^{n-1} 1 \times \Delta x}=\frac{\int_{a}^{b}f(x)dx}{\int_{a}^{b}1 dx}=\frac{1}{b-a} \int_{a}^{b}f(x)dx\]
\end{enumerate}


\end{document} 