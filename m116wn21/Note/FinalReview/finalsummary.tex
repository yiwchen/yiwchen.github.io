\documentclass[12pt]{article}
\usepackage{fullpage}
\usepackage{amsthm}
\usepackage{amsfonts,amsmath, amssymb,latexsym,mathrsfs}
\usepackage[margin=1.15in]{geometry}
\usepackage{enumitem}
\setlength{\parindent}{0pt}
\usepackage{tikz-cd}
\usepackage{fancyhdr}




\theoremstyle{definition}
\newtheorem{thm}{Theorem}[section]

\newtheorem{prop}{Proposition}[section]


\theoremstyle{definition}
\newtheorem{definition}{Definition}[section]

\theoremstyle{remark}
\newtheorem*{remark}{Remark}

\theoremstyle{definition}
\newtheorem{example}{Example}[section]


\theoremstyle{definition}
\newtheorem{lem}{Lemma}[section]


\theoremstyle{definition}
\newtheorem{cor}{Corollary}[section]


\date{}
\title{Final Exam Summary(After Exam 2)}


\begin{document}
\maketitle

\section{Power Series}
\begin{definition}
	A power series about $x = a$ is a sum of constants times powers of $(x - a)$: 
	$C_0 + C_1(x - a) + C_2(x - a)^2 + \ldots + C_n(x - a)^n + \ldots =	\sum_{n
	=0}^{\infty}	C_n(x - a)^n$.
\end{definition}

If we fix a specific value of $x$, we can just consider plugging x with the value we have, and convergence here makes sense.

\begin{definition}
For a fixed value of $x$, if this sequence of partial sums converges to a limit $L$, that is, if
$\lim_{n \to \infty}S_n(x) = L$, then we say that the power series converges to $L$ for this value of $x$.
\end{definition}

Based on the discussion we will see that, The interval of convergence for a power series is usually centered at a point $x=a$, and extends the same length to both side, thus we denote this length as radius of convergence.\\

Moreover, each power series falls into one of the three following cases, characterized by its \textcolor{red}{radius of convergence}, $R$.
\begin{itemize}
\item The series converges only for $x = a$; the radius of convergence is defined to be $R = 0$.
\item The series converges for all values of $x$; the radius of convergence is defined to be
$R = \infty$.
\item There is a positive number $R$, called the radius of convergence, such that the series
converges for $|x - a| < R$ and diverges for $|x - a| > R$. 
\end{itemize}
The interval of convergence is the interval between $a - R$ and $a + R$, including any
endpoint where the series converges.
\\

Then there is a question arises, how to find this radius of convergence then?\\

This question can be determined by considering using \textcolor{red}{ratio test} on the series, assuming $x\neq a$.
The details are included in Chapter 9.5 in the book.

\section{Taylor Polynomial and Taylor Series}
\subsection{Taylor Polynomial}
If we try to approximate the function locally using a polynomial, there is one thing we want to acquire, i.e. we want the polynomial $P(x)$ with the property that $P^{(n)}(a)=f^{(n)}(a)$ if we approximate the function at the point $x=a$. Considering merely the situation about $x=0$, recall what we did in the class, we will have the following.

\textcolor{red}{Taylor Polynomial of Degree $n$ Approximating $f(x)$ for $x$ near $0$} is \[f(x) \approx P_n(x)
	= f(0) + f'(0)x + \frac{f''(0)}{2!}x^2 + \frac{f'''(0)}{3!}x^3 + \frac{f^{(4)}(0)}{4!} x^4 + \ldots + \frac{f^{(n)}(0)}{n!} x^n\]
	We call $P_n(x)$ the Taylor polynomial of degree $n$ centered at $x = 0$, or the Taylor poly
	nomial about $x = 0$.\\
	
More generally, \textcolor{red}{Taylor Polynomial of Degree $n$ Approximating $f(x)$ for $x$ near $a$} is \[f(x) \approx P_n(x)
= f(a) + f'(a)(x-a) + \frac{f''(a)}{2!}(x-a)^2 + \frac{f'''(a)}{3!}(x-a)^3  + \ldots + \frac{f^{(n)}(a)}{n!} (x-a)^n\]
We call $P_n(x)$ the Taylor polynomial of degree $n$ centered at $x = a$, or the Taylor poly
nomial about $x =a$.\\

Notice that Taylor Polynomial of Degree $n$ Approximating $f(x)$ for $x$ near $a$ will have the property that $P_n^{(m)}(a)=f^{(m)}(a)$ for $0 \leq m \leq n$.

\subsection{Taylor Series}

Notice that in the Taylor polynomial, if we let $n$ here goes to infinity, we will get a series $P(x)$ with $P^{(m)}(a)=f^{(m)}(a)$ for $0 \leq m < \infty$ and thus we will expect that the series gives a good approximation about $f(x)$ around $a$, and actually when it converges, it is exactly the value you will get in $f(x)$, and this is called the Taylor Series.

\textcolor{red}{Taylor Series for $f(x)$ about $x=0$} is \[f(x) = f(0) + f'(0)x + \frac{f''(0)}{2!}x^2 + \frac{f'''(0)}{3!}x^3 + \frac{f^{(4)}(0)}{4!} x^4 + \ldots + \frac{f^{(n)}(0)}{n!} x^n+ \ldots \]
We call $P_n(x)$ the Taylor polynomial of degree $n$ centered at $x = 0$, or the Taylor poly
nomial about $x = 0$.\\

More generally, \textcolor{red}{Taylor Series for $f(x)$ about $x=a$} is \[f(x) = f(a) + f'(a)(x-a) + \frac{f''(a)}{2!}(x-a)^2 + \frac{f'''(a)}{3!}(x-a)^3  + \ldots + \frac{f^{(n)}(a)}{n!} (x-a)^n+ \ldots \]
We call $P_n(x)$ the Taylor polynomial of degree $n$ centered at $x = a$, or the Taylor poly
nomial about $x =a$.\\


Moreover, there are \textcolor{red}{several important cases} that we consider, each of them is an Taylor expansion of a function about $x=0$:
\begin{itemize}
\item \[e^{x}= 1 + x + \frac{x^2}{2!} + \frac{x^3}{3!} + \frac{x^4}{4!} + \frac{x^5}{5!} + \frac{x^6}{6!} + \frac{x^7}{7!} + \frac{x^8}{8!} + \cdots\text{ converges for all } x\]
\item \[\sin(x)=\sum\limits_{n=0}^\infty \dfrac{x^{2n+1}}{(2n+1)!}\cdot(-1)^n = x-\dfrac{x^3}{3!}+\dfrac{x^5}{5!}-\dfrac{x^7}{7!}+\dots\text{ converges for all } x\]
\item \[\cos(x)=\sum\limits_{n=0}^\infty \dfrac{x^{2n}}{(2n)!}\cdot(-1)^n = 1-\frac{x^2}{2!}+\frac{x^4}{4!}-\frac{x^6}{6!}+\dots \text{ converges for all } x\]
\item \[(1 + x)^p = \sum_{k=0}^{\infty} \binom{p}{k} x^k= \sum_{k=0}^{\infty} \frac{p!}{k!(p-k)!} x^k=\]\[1 + px + \frac{p(p - 1)}{2!}x^2 + \frac{p(p - 1)(p - 2)}{3!}x^3 + \cdots \text{ converges for } -1 < x < 1.\]
\item \[\ln(1+x) =\sum_{n = 0}^{\infty}\frac{(-1)^nx^{n+1}}{n+1}= x-\frac{x^2}{2}+\frac{x^3}{3}-\frac{x^4}{4}+\cdots,\]

\end{itemize}

Moreover, we can definitely find Taylor Series based on the existing series using \textcolor{red}{four methods}:
\begin{itemize}
	\item Substitude
	
Example: Taylor Series about $x=0$ for $f(x)=e^{-x^2}$	
	
	\item Differentiate 
	
Example: Taylor Series about $x=0$ for $f(x)=\frac{1}{(1-x)^2}$	

	\item Integrate
	
Example: Taylor Series about $x=0$ for $f(x)=\arctan x$ (Hint: What is $\frac{d}{dx}(\arctan x)$?)			
	\item Multiply
	
	Example: Taylor Series about $x=0$ for $f(x)=x^2 \sin x$\\
	Example: Taylor Series about $x=0$ for $f(x)=\sin x \cos x$\\
	Example: Taylor Series about $x=0$ for $f(x)=e^{\sin x}$\\

\end{itemize}

\section{Parametric Equations and Polar Coordinate}
\subsection{Parametric Equations}
To represent the motion of a particle in the $xy$-plane we use two equations, $x=f(t)$ and $y=g(t)$, then at the time $t$ the particle is at the location $(f(t),g(t)$. In this case, we call the equations for $x$ and $y$ the parametric equations, with parametrization $t$.

Remember that, in parametric equation, for the same line, the parametrization is not unique, and the different parametrization encodes two information:\\
1. Speed of the particle.\\
2. Direction of the motion.\\

\subsubsection{Special Parametric Equations}

\begin{itemize}
\item \textcolor{red}{Parametric Equations for a Straight Line}

An object moving along a line through the point $(x_0, y_0)$, with $dx/dt = a$ and $dy/dt = b$,
has parametric equations
$x = x_0 + at, y = y_0 + bt$.
The slope of the line is $m = b/a$.
\item\textcolor{red}{Parametric Equations for a circle with radius $k$}

An object moving along a circle of radius $k$ counterclockwise has parametric equations
$x = k\cos(t), y = k\sin(t)$.
\end{itemize}


\subsubsection{Slope and concavity of the curve}
As we discussed in class, we can think of this as a result due to chain rule if we have that $y=F(x)$ as well. 

But to summarize, we have the \textcolor{red}{slope} of the parametrized curve to be 
\[\frac{dy}{dx}=\frac{dy/dt}{dx/dt}\]
and the \textcolor{red}{concavity} of the parametrized curve to be
\[\frac{d^2y}{dx^2}=\frac{(dy/dx)/dt}{dx/dt}\]

\subsubsection{Speed and distance}

The \textcolor{red}{instantaneous speed} of a moving object is defined to be
$$v = \sqrt{(dx/dt)^2 + (dy/dt)^2} =\sqrt{(v_x)^2 + (v_y)^2}$$.
The quantity $v_x = dx/dt$ is the instantaneous velocity in the $x$-direction; $v_y = dy/dt$ is the
instantaneous velocity in the $y$-direction.
And we call that $(v_x,v_y)$ to be the velocity vector.

Moreover, the \textcolor{red}{distance} traveled from time $a$ to $b$ is $$\int^b_a v(t) dt = \int_a^b \sqrt{(dx/dt)^2 + (dy/dt)^2} dt$$

\subsection{Polar Coordinate}

Polar coordinates is the coordinates determined by specifying the distance of the point to origin and the angle measured counterclockwise from positive $x$-axis to the line joining the line connecting the point and the origin.

\subsubsection{Relation between Cartesian and Polar}

\textcolor{red}{Cartesian to Polar}: $$(x,y) \to (r= \sqrt{x^2 + y^2}, \theta) \text{ (Here we have that } \tan \theta = \frac{y}{x} \text{)}$$
Note that $\theta$ does not have to be $\arctan(\frac{y}{x})$!

Polar to Cartesian: $$(r,\theta) \to (x=r \cos \theta, y=r \sin \theta)$$

\subsubsection{Slope, Arc length and Area in Polar Coordinates}

By the relation $x=r \cos \theta, y=r \sin \theta$, given a curve $r=f(\theta)$, we have that $x=f(\theta) \cos \theta, y=f(\theta) \sin \theta$, and thus are parametrized equations of parameter $\theta$. Therefore we have that the \textcolor{red}{slope} of to be 
\[\frac{dy}{dx}=\frac{dy/d\theta}{dx/d\theta}\]

The \textcolor{red}{arc length} from angle $a$ to $b$ is $$\int_a^b \sqrt{(dx/d\theta)^2 + (dy/d\theta)^2} d\theta=\int_a^b \sqrt{r^2 + (dr/d\theta)^2} d\theta$$

Moreover, due to the fact that the \textcolor{red}{area of the sector} is $1/2 r^2 \theta$, we have that for a curve $r = f(\theta)$, with \textcolor{red}{$f(\theta)$ continuously of the same sign}, the area of the region enclosed is $$\frac{1}{2}\int^{b}_{a}f(\theta)^2 d\theta$$


\end{document} 