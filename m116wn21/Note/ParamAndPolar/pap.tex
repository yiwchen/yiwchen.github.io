\documentclass[12pt]{article}
\usepackage{fullpage}
\usepackage{amsthm}
\usepackage{amsfonts,amsmath, amssymb,latexsym,mathrsfs}
\usepackage[margin=1.15in]{geometry}
\usepackage{enumitem}
\setlength{\parindent}{0pt}
\usepackage{tikz-cd}
\usepackage{fancyhdr}




\theoremstyle{definition}
\newtheorem{thm}{Theorem}[section]

\newtheorem{prop}{Proposition}[section]


\theoremstyle{definition}
\newtheorem{definition}{Definition}[section]

\theoremstyle{remark}
\newtheorem*{remark}{Remark}

\theoremstyle{definition}
\newtheorem{example}{Example}[section]


\theoremstyle{definition}
\newtheorem{lem}{Lemma}[section]


\theoremstyle{definition}
\newtheorem{cor}{Corollary}[section]


\date{}
\title{Parametrization and Polar Coordinates}


\begin{document}
\maketitle
\section{Parametric Equations and Polar Coordinate}
\subsection{Parametric Equations}
To represent the motion of a particle in the $xy$-plane we use two equations, $x=f(t)$ and $y=g(t)$, then at the time $t$ the particle is at the location $(f(t),g(t)$. In this case, we call the equations for $x$ and $y$ the parametric equations, with parametrization $t$.

Remember that, in parametric equation, for the same line, the parametrization is not unique, and the different parametrization encodes two information:\\
1. Speed of the particle.\\
2. Direction of the motion.\\

\subsubsection{Special Parametric Equations}

\begin{itemize}
\item \textcolor{red}{Parametric Equations for a Straight Line}

An object moving along a line through the point $(x_0, y_0)$, with $dx/dt = a$ and $dy/dt = b$,
has parametric equations
$x = x_0 + at, y = y_0 + bt$.
The slope of the line is $m = b/a$.
\item\textcolor{red}{Parametric Equations for a circle with radius $k$}

An object moving along a circle of radius $k$ counterclockwise has parametric equations
$x = k\cos(t), y = k\sin(t)$.
\end{itemize}


\subsubsection{Slope and concavity of the curve}
As we discussed in class, we can think of this as a result due to chain rule if we have that $y=F(x)$ as well. 

But to summarize, we have the \textcolor{red}{slope} of the parametrized curve to be 
\[\frac{dy}{dx}=\frac{dy/dt}{dx/dt}\]
and the \textcolor{red}{concavity} of the parametrized curve to be
\[\frac{d^2y}{dx^2}=\frac{(dy/dx)/dt}{dx/dt}\]

\subsubsection{Speed and distance}

The \textcolor{red}{instantaneous speed} of a moving object is defined to be
$$v = \sqrt{(dx/dt)^2 + (dy/dt)^2} =\sqrt{(v_x)^2 + (v_y)^2}$$.
The quantity $v_x = dx/dt$ is the instantaneous velocity in the $x$-direction; $v_y = dy/dt$ is the
instantaneous velocity in the $y$-direction.
And we call that $(v_x,v_y)$ to be the velocity vector.

Moreover, the \textcolor{red}{distance} traveled from time $a$ to $b$ is $$\int^b_a v(t) dt = \int_a^b \sqrt{(dx/dt)^2 + (dy/dt)^2} dt$$

\subsection{Polar Coordinate}

Polar coordinates is the coordinates determined by specifying the distance of the point to origin and the angle measured counterclockwise from positive $x$-axis to the line joining the line connecting the point and the origin.

\subsubsection{Relation between Cartesian and Polar}

\textcolor{red}{Cartesian to Polar}: $$(x,y) \to (r= \sqrt{x^2 + y^2}, \theta) \text{ (Here we have that } \tan \theta = \frac{y}{x} \text{)}$$
Note that $\theta$ does not have to be $\arctan(\frac{y}{x})$!

\textcolor{red}{Polar to Cartesian}: $$(r,\theta) \to (x=r \cos \theta, y=r \sin \theta)$$

\subsubsection{Slope, Arc length and Area in Polar Coordinates}

By the relation $x=r \cos \theta, y=r \sin \theta$, given a curve $r=f(\theta)$, we have that $x=f(\theta) \cos \theta, y=f(\theta) \sin \theta$, and thus are parametrized equations of parameter $\theta$. Therefore we have that the \textcolor{red}{slope} of to be 
\[\frac{dy}{dx}=\frac{dy/d\theta}{dx/d\theta}\]

The \textcolor{red}{arc length} from angle $a$ to $b$ is $$\int_a^b \sqrt{(dx/d\theta)^2 + (dy/d\theta)^2} d\theta$$

Moreover, due to the fact that the \textcolor{red}{area of the sector} is $1/2 r^2 \theta$, we have that for a curve $r = f(\theta)$, with \textcolor{red}{$f(\theta) \geq 0$}, the area of the region enclosed is $$\frac{1}{2}\int^{b}_{a}f(\theta)^2 d\theta$$

\end{document} 