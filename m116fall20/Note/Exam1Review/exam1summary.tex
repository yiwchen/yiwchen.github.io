\documentclass[12pt]{article}
\usepackage{fullpage}
\usepackage{amsthm}
\usepackage{amsfonts,amsmath, amssymb,latexsym,mathrsfs}
\usepackage[margin=1.15in]{geometry}
\usepackage{enumitem}
\setlength{\parindent}{0pt}
\usepackage{tikz-cd}
\usepackage{fancyhdr}




\theoremstyle{definition}
\newtheorem{thm}{Theorem}[section]

\newtheorem{prop}{Proposition}[section]


\theoremstyle{definition}
\newtheorem{definition}{Definition}[section]

\theoremstyle{remark}
\newtheorem*{remark}{Remark}

\theoremstyle{definition}
\newtheorem{example}{Example}[section]


\theoremstyle{definition}
\newtheorem{lem}{Lemma}[section]


\theoremstyle{definition}
\newtheorem{cor}{Corollary}[section]


\date{}
\title{Exam 1 summary}


\begin{document}
\maketitle




\section{Finding Antiderivative, introduction}

\begin{definition}
	A definite integral of $f$ from $a$ to $b$ is definted as 
\[\int_a^bf(x)dx= \lim_{n\rightarrow \infty} \sum_{i=1}^n f(x_i)\Delta x\text{\bf (Limit of Right-hand sum)}\]
or 
\[\int_a^bf(x)dx= \lim_{n\rightarrow \infty} \sum_{i=0}^{n-1} f(x_i)\Delta x\text{\bf (Limit of Left-hand sum)}\]

Here, Left-hand sum and Right-hand sum are equal after taking limits and it the the so-called Riemann Sum.
\end{definition}



There are two more type of Riemann sum I would like to discuss in the future, which is the Mid sum and the Trapezoidal sum. 

I will only give definition here.
\[\int_a^bf(x)dx\approx  \sum_{i=0}^{n-1} f(\frac{x_i+x_{i+1}}{2})\Delta x\text{\bf (Mid sum)}\]
and
\[\int_a^bf(x)dx\approx \sum_{i=0}^{n-1} \frac{f(x_i)+f(x_{i+1})}{2}\Delta x\text{\bf (Trapezoid sum)}\]

In all these Riemann sum we discussed, we are assuming $\Delta(x)=\dfrac{b-a}{n}$, thus as $n\to \infty$, $\Delta x \to 0$.

Note that \[\frac{LEFT(n)+RIGHT(n)}{2}=TRAP(n)\], \[MID(n)\neq TRAP(n)\].

\newpage

\begin{remark}
	Properties of definite integral:
	\begin{enumerate}
		\item \[\int^a_b f(x) dx = -\int^b_a f(x) dx\]
		\item \[\int^a_b f(x) dx+\int^b_c f(x) dx=\int^a_c f(x) dx\]
		\item \[\int^a_b (f(x)\pm g(x)) dx=\int^a_b f(x) dx \pm \int^a_b g(x) dx\]
		\item \[\int^a_b cf(x) dx = c \int^a_b f(x) dx\]
		\item Symmetry due to the oddity of the function.
	\end{enumerate}
\end{remark}

\hrulefill

\begin{remark}
	Interpretation of Define Integral as Area under graph of $f$ between $x=a$ and $x=b$, counting positivity.
\end{remark}

Important cases discussed on Friday's course:
\begin{enumerate}
	\item Compute \[\int^{1}_{-1}\sqrt{1-x^2}dx\]
	\item How about \[\int^{1}_{-1}(\sqrt{1-x^2}-1)dx?\]
	\item Maybe try\[\int^{0.5}_{-0.5}\tan(x)dx\]
\end{enumerate}

Find out the answer yourself only geometrically, even you know more techniques!



More Importantly there are two major topics I want to mention here and maybe discuss:

\begin{itemize}
	\item When is the estimation done by Riemann sum a underestimate/overestimate?
	
\textbf{It is also covered in 7.5, check it out and try problem 2.}
	
	\item Error estimation
	
	Think about the case where you know that $f(x)$ lies between any pair of $LEFT(n)$ and $RIGHT(n)$, then we see that $|LEFT(n) - f(x)|<|LEFT(n)-RIGHT(n)|=(f(b)-f(a))\Delta x$. This usually gives a bound for $n$.
\end{itemize}

\hrulefill
\begin{thm}
\textbf{The Fundamental Theorem of Calculus} is basically the theorem defined below.

If $f$ is continuous on interval $[a,b]$ and $f(t)=F'(t)$, then \[\int^b_a f(t) dt = F(b)-F(a).\]
\end{thm}
\hrulefill

\textbf{Application of Definite Integral}

\begin{enumerate}
	\item Average value of function $f(x)$ in $[a,b]$ is \[\frac{1}{b-a} \int_{a}^{b}f(x)dx\].
	
	Here, think about the integration is analogue to summation in the discrete world, then this average value is the analogue of \[x_{\text{average}}=\frac{x_1+x_2+\ldots+x_n}{1+1+1+\dots}=\frac{x_1+x_2+\ldots+x_n}{n}.\] where in the integration case is actually \[x_{\text{average}}=\frac{\lim_{n\rightarrow \infty} \sum_{i=0}^{n-1} f(x_i)\Delta x}{\lim_{n\rightarrow \infty} \sum_{i=0}^{n-1} 1 \times \Delta x}=\frac{\int_{a}^{b}f(x)dx}{\int_{a}^{b}1 dx}=\frac{1}{b-a} \int_{a}^{b}f(x)dx\]
\end{enumerate}

\textbf{Anti-derivative of usual functions}

\begin{enumerate}
	\item Try to find the antiderivatives by the graphs
	\item Compute an antiderivative using definite integrals.
\end{enumerate}

Suggested Problems: $\mathsection$ 6.1 3,7,9,13, 17,29,31,33

\hrulefill

\textbf{Construct antiderivative analytically}

\begin{definition}
	We define the general antiderivative family as indefinite integral.
\end{definition}

\begin{remark}
	\[\int C dx = 0\]
	\[\int kdx=kx+C\]
	\[\int x^ndx=\frac{x^{n+1}}{n+1}+C, (n \neq -1)\]
	\[\int \frac{1}{x}dx = \ln|x|+C\]
	\[\int e^xdx=e^x+C \]
	\[\int \cos xdx=\sin x + C \]
	\[\int \sin xdx=-\cos x + C \]	
	
\end{remark}


\hrulefill


Properties of antiderivatives:

\begin{enumerate}
\item \[\int (f(x) \pm g(x))dx=\int f(x) dx \pm \int g(x) dx\]
\item \[\int cf(x) dx = c \int f(x) dx\]
\end{enumerate}


Suggested Problems: $\mathsection$  6.2 51-59, 65,71,75


\hrulefill

\textbf{Second FTC (Construction theorem for Antiderivatives)}

\begin{thm}
If $f$ is a continuous function on an interval, and if $a$ is any number in that interval then the function $F$ defined on the interval as follows is an antiderivative of $f$:

\[F(x)=\int^x_a f(t) dt\]
\end{thm}

Suggested Problems: $\mathsection$ 6.4  5,7,9,11,17,27,31-34



\section{Integration Techniques}

In this section, we will focus more on how to find the family of antiderivatives that is determined by the function we provided, since if we know such family, then given an initial value of $F(x)$, we can recover the specific antiderivative that we need.

\subsection{Guess and Check method}

Frankly speaking, this is gambling based on experience.\\
Although it is one of the fundamental method we use to solve problems, for example, it appears in also cryptography with the name Bruce force attack, and it also appears in computer science with the name Brute force search. However, as you probably know, this is usually not the most systematic or the most efficient way to the solution, and it usually requires either experience on the work or machine that do not rest. 

\subsection{Integration using Substitution}
Exercise as motivation: \textsection 6.4 35,37.

Similarly, the chain rule of derivative is also extremely useful in finding integral.

Theoretically, integration using substitution is using the philosophy that given an integration as $\int f(g(x)) g'(x) dx$, we will have $F(g(x))+C$ as its derivative, which is due to $\frac{d}{dx}(F(g(x))) = f(g(x))g'(x)$, and doing integration on both side will give us the ''substitution rule'' we want.

Technically, you want to recognize the ''$g(x)$'' part in your function, then substitude $g(x)$ as $w$, with the rule $dw=w'(x)dx =\frac{dw}{dx} dx$ (naively, we do have $\frac{dw}{dx} dx =dw$! However, this does not make much sense here which if you are particular interesting, you can ask me in person.)

Now let us consider an example, to see how is it actually applies.
Personally, my favorite almost trivial example is \[\int^{\infty}_0 e^{-x}dx=-\int_{0}^{-\infty}e^w -dw=\int^{0}_{-\infty}e^w dw=1.\text{ (Here I let }w=-x)\]

I will explain why is this one interesting in the next subsection, but here there is a warning you may see, \textbf{Check the up and low bound for your integration!} Let us see another example to see this clearer.

\[\int_{0}^{\pi/4}\frac{\tan^3 \theta}{\cos^2 \theta}d \theta = \int_{0}^{1}w^3dw=\frac{1}{4},\text{ ( Here I take } w = \cos \theta.)\]

Fast question: Can I do the following integrals? Why?$$(1) \int_{0}^{\pi/2}\frac{\tan^3 \theta}{\cos^2 \theta}d \theta  \text{ or } (2)\int_{-\pi/4}^{\pi/4}\frac{\tan^3 \theta}{\cos^2 \theta}d \theta$$


Suggested Problems: $\mathsection$7.1 some ex, 81,89,90-96,97,99,109,111,114,125,141,145,147


\subsection{Integration using partial fractions}

Small break after the introduction to substitution, we have this partial fractions.

Using the method of the substitution, we can already compute many weird integration like $\int \frac{2}{x-10} dx$.

Now, how about $\int \frac{1}{x^2-1} dx$? We are out of knowledge about how to do it right now, but there is an important method we ususally can apply, which is the partial fractions. 

The idea here is to break $\frac{1}{x^2-1}$ as a sum: $$\frac{1}{x^2-1}=\frac{1}{2(x-1) }+(-\frac{1}{2(x+1)})$$

But then now we can compute the integration in the question, since we can definitely compute $\int \frac{1}{2(x-1) } dx$ and $\int \frac{1}{2(x+1) }dx$.

If you run into repeated factors? The most general case (example): $$\frac{p(x)}{(x+c_1)^2(x+c_2)(x^2+c_3)}=\frac{A}{x+c_1}+\frac{B}{(x+c_1)^2}+\frac{C}{x+c_2}+ \frac{Dx+E}{x^2+c_3}$$

\subsection{Integration by parts}
If we call the integration using substitution the masterpiece established on the Chain Rule, then I have to say integration by parts is the same thing established on product rule.

Similar to the substitution, we reverse our thoughts here.

What is exactly happen in the integration by parts is that we are just undoing the product rules. Therefore, the key part is recognize your product rule carefully, 

Consider the product rule:
\[(uv)'=u'v+uv'=u'v+uv' \]
By some twisted, we will have the following formula for $uv'$ which we have not seen the advantages yet.
\[uv'=(uv)'−u'v\]
But now, integrate both sides will give us:
\[\int udv=\int(uv)′−\int vdu= uv - \int v du\], now if you remember what we talked about in the substitution chapter, you will immediately recognize that this is what we want.

Traditional Example:\[\int \ln(x) dx\] \[\int \cos^2 x\]

Suggested Problems: $\mathsection$ 7.2 49,53,55,65,69,73

\newpage

\section{Integration Approximation}

Estimate the integral is based on the fact that after taking limits of Left/Right/Mid/Trapezoidal Riemann Sum, they are gonna equals and will give the actual integration value, thus if we just compute the Riemann sums in finite many block, we will have a value that is pretty much close to the actual integration, and we can refine our calculation by taking more blocks.

Reminder: we have then four ways of estimating an integral using a Riemann Sum:
\begin{enumerate}
\item LEFT(n)
\item RIGHT(n)
\item MID(n)
\item TRAP(n)
\end{enumerate}
Think about what they are?
\textbf{Think about Pictures!}

Is your estimation over/under?\\

\textbf{Again, Pictures!}(It probably will never make sense without picture.)
\begin{enumerate}
\item If the graph of $f$ is increasing on $[a,b]$, then $$LEFT(n)\leq \int^b_a f(x) dx \leq RIGHT(n)$$
\item If the graph of $f$ is decreasing on $[a,b]$, then $$RIGHT(n)\leq \int^b_a f(x) dx \leq LEFT(n)$$
\item If the graph of $f$ is concave up on $[a,b]$, then $$MID(n)\leq \int^b_a f(x) dx \leq TRAP(n)$$
\item If the graph of $f$ is concave down on $[a,b]$, then $$TRAP(n)\leq \int^b_a f(x) dx \leq MID(n)$$

\end{enumerate}

Suggested Problems: $\mathsection$7.5	1,3,5,16,17,25



\section{Find Area/Volumes by slicing}
\begin{itemize}
	\item Compute the area of triangle;
	\item Compute the area of (semi)circle;
	\item Compute the volume of a sphere; 
	\item Compute the volume of a cone with base radius $5$ and height $5$; 
	\item Volume of revolution: $y = e^{-x}$ from $0$ to $1$ around $x$-axis.
\end{itemize}

Suggested Problems: $\mathsection$8.1	1-4,10,12,14,16,18,19,21,29,31,37

\section{Volumes of Solids of Revolution}
There are two methods to find such a volume of solids of revolution(here revolution is important as it brings in the symmetries).
\subsection{Disk Method}

 One way is using the slicing which is very similar to the 2D case when we try to find the area. This is officially called the Disk Method.
Check above.

\subsection{Shell Method}

This is not so intuitive that why will we want to consider other method to integrating any function since we already have this disk method. However, consider the following example:

Determine the volume of the solid obtained by rotating the region bounded by $y=(x-1)(x-3)^2$ and the $x$-axis about the $y$-axis.

Try to graph it and see why it is bad.

Thus, the shell method arises naturally as a different approach, as it integrates along an axis perpendicular to the axis of revolution.

\section{Arc length}

\[\text{Arc Length}=\int^b_a\sqrt{1+(f'(x))^2}dx\]

The function we are integrating might not see as intuitive, but when we think about the picture of ''small increment'' as in class, you will see how it is making sense.

Suggested Problems: $\mathsection$8.2 25,27,35,47,49,57,63,65


\section{Relation of Integration and Physics}

Think about what integration is doing. Integral is defined as a limit of the Riemann sum, thus what it is doing is exactly the same as what the Riemann sum is doing, furthermore, you can think definite integral as just infinite block Riemann sum.\\
What we had done in the last section is to find Volumes by integrating, which is the first time we figure out what this mysterious ''f(x)'' in the integration really is, i.e. if it is an area, then $f(x)\Delta x$ is actually a volume, and we are summing many slices/shells of volumes to get the exact amount of volume of the object.\\
Here, the most essential formula we are using is actually $V=Ah$, where $V$ is the volume of prism/cylinder, $A$ is the area, and $h$ is the height. In the most settings above, we will see $S$ as function of $h$(which is exactly disk method), or sometimes $h$ is a function of $A$( what we only deal are the good cases, where $S$ is a function of $r$ and $h$ is a function of  $r$, since $dS$ is not making sense yet).

\subsection{Mass}


Here, the basic formula we are doing is:\begin{enumerate}
\item One dimensional: $M=\delta l$ where $M$ is the total mass, $\delta$ is the density, $l$ is line.
\item
Two dimensional: $M=\delta A$ where $M$ is the total mass, $\delta$ is the density, $A$ is Area.
\item Three dimensional (real world): $M=\delta V$ where $M$ is the total mass, $\delta$ is the density, $V$ is Volume.
\end{enumerate}

It is confusing that when should one use which, but if you think about what the UNIT of the density is, it will make sense of itself. 

(Alternative way to think of integration: actually, if you think area as the density of Volume, i.e. think of unit of it as $m^2=m^3/m$, then you will see integration is basically integrating the density to get the total, this is actually the essential picture you should have. In the end of day, they are all just sum.)


\subsection{Work}

Key formula we are using:\\\[
\text{Work done = Force} \cdot \text{Distance}\] or 
\[W=F\cdot s\]

Integration version:
\[W = \int^b_a F(x) dx\]
(Question: why usually not $x = x(F)$?)



\end{document} 