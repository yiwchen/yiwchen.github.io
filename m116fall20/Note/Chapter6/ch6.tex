\documentclass[12pt]{article}
\usepackage{fullpage}
\usepackage{amsthm}
\usepackage{amsfonts,amsmath, amssymb,latexsym,mathrsfs}
\usepackage[margin=1.15in]{geometry}
\usepackage{enumitem}
\setlength{\parindent}{0pt}
\usepackage{tikz-cd}
\usepackage{fancyhdr}




\theoremstyle{definition}
\newtheorem{thm}{Theorem}[section]

\newtheorem{prop}{Proposition}[section]


\theoremstyle{definition}
\newtheorem{definition}{Definition}[section]

\theoremstyle{remark}
\newtheorem*{remark}{Remark}

\theoremstyle{definition}
\newtheorem{example}{Example}[section]


\theoremstyle{definition}
\newtheorem{lem}{Lemma}[section]


\theoremstyle{definition}
\newtheorem{cor}{Corollary}[section]


\date{}
\title{Chapter 6: Finding Antiderivative, introduction}


\begin{document}
\maketitle





\textbf{Anti-derivative of usual functions}

\begin{enumerate}
	\item Try to find the antiderivatives by the graphs
	\item Compute an antiderivative using definite integrals.
\end{enumerate}

Suggested Problems: $\mathsection$ 6.1 3,7,9,13, 17,29,31,33

\hrulefill

\textbf{Construct antiderivative analytically}

\begin{definition}
	We define the general antiderivative family as indefinite integral.
\end{definition}

\begin{remark}
	\[\int C dx = 0\]
	\[\int kdx=kx+C\]
	\[\int x^ndx=\frac{x^{n+1}}{n+1}+C, (n \neq -1)\]
	\[\int \frac{1}{x}dx = \ln|x|+C\]
	\[\int e^xdx=e^x+C \]
	\[\int \cos xdx=\sin x + C \]
	\[\int \sin xdx=-\cos x + C \]	
	
\end{remark}


\hrulefill


Properties of antiderivatives:

\begin{enumerate}
\item \[\int (f(x) \pm g(x))dx=\int f(x) dx \pm \int g(x) dx\]
\item \[\int cf(x) dx = c \int f(x) dx\]
\end{enumerate}


Suggested Problems: $\mathsection$  6.2 51-59, 65,71,75


\hrulefill

\textbf{Second FTC (Construction theorem for Antiderivatives)}

\begin{thm}
If $f$ is a continuous function on an interval, and if $a$ is any number in that interval then the function $F$ defined on the interval as follows is an antiderivative of $f$:

\[F(x)=\int^x_a f(t) dt\]
\end{thm}

Suggested Problems: $\mathsection$ 6.4  5,7,9,11,17,27,31-34



\end{document} 