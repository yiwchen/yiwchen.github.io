\documentclass[12pt]{article}
\usepackage{fullpage}
\usepackage{amsthm}
\usepackage{amsfonts,amsmath, amssymb,latexsym,mathrsfs}
\usepackage[margin=1.15in]{geometry}
\usepackage{enumitem}
\setlength{\parindent}{0pt}
\usepackage{tikz-cd}
\usepackage{fancyhdr}




\theoremstyle{definition}
\newtheorem{thm}{Theorem}[section]

\newtheorem{prop}{Proposition}[section]


\theoremstyle{definition}
\newtheorem{definition}{Definition}[section]

\theoremstyle{remark}
\newtheorem*{remark}{Remark}

\theoremstyle{definition}
\newtheorem{example}{Example}[section]


\theoremstyle{definition}
\newtheorem{lem}{Lemma}[section]


\theoremstyle{definition}
\newtheorem{cor}{Corollary}[section]


\date{}
\title{Chapter 8: Application of integral in real life}


\begin{document}
\maketitle

\section{Probability}
\subsection{PDF and CDF}
\begin{definition}
A function $p(x)$ is a \textbf{probability density function} or PDF if it satisfies the following conditions
\begin{itemize}
\item $p(x) \geq 0$ for all $x$.
\item $\int_{-\infty}^\infty p(x) = 1.$
\end{itemize}
\end{definition}

\begin{definition}
A function $P(t)$ is a \textbf{Cumulative Distribution Function} or cdf, of a density function $p(t)$, is defined by 
\[P(t) =\int_{-\infty}^t p(x) dx \]
Which means that $P(t)$ is the antiderivative of $p(t)$ with the following properties:
\begin{itemize}
\item $P(t)$ is increasing and $0\leq P(t)\leq 1$ for all $t$.
\item $\lim_{t \to \infty}P(t)=1.$
\item $\lim_{t \to -\infty}P(t)=0.$
\end{itemize}
\end{definition}

Moreover, we have $\int_a^b p(x)dx=P(b)-P(a)$.

\subsection{Probability, mean and median}

\subsubsection*{Probability}
Let us denote $X$ to be the quantity of outcome that we care ($X$ is in fact, called the random variable).
\[\mathbb{P}\{a\leq X\leq b\}=\int_a^b p(x)dx=P(b)-P(a)\]
\[\mathbb{P}\{X\leq t\}=\int_{-\infty}^t p(x)dx=P(t)\]
\[\mathbb{P}\{X\geq s\}=\int_{s}^\infty p(x)dx=1-P(s)\]

\subsubsection*{The mean and median}
\begin{definition}
A \textbf{median} of a quantity $X$ is a value $T$ such that the probability of $X\leq T$ is $1/2$. Thus we have  $T$ is defined by the value such that
\[ \int_{-\infty}^T p(x) dx=1/2 \] or \[P(T)=1/2\].
\end{definition}
\begin{definition} A \textbf{mean} of a quantity $X$ is the value given by
	\[ Mean= \frac{\text{Probability of all possible quantity}}{\text{Total probability}}= \frac{\int_{-\infty}^{\infty}xp(x)dx}{\int_{-\infty}^{\infty}p(x)dx}=\frac{\int_{-\infty}^{\infty}xp(x)dx}{1}=\int_{-\infty}^{\infty}xp(x)dx. \]
\end{definition}

\subsubsection*{Normal Distribution}
\begin{definition}
A normal distribution has a density function of the form 
\[p(x)=\frac{1}{\sigma \sqrt{2 \pi}}e^{-\frac{(x-\mu)^2}{2\sigma^2}}\] where $\mu$ is the mean of the distribution and $\sigma$ is the standard deviation, with $\sigma > 0$.
The case $\mu = 0$, $\sigma = 1$ is called the standard normal distribution.

\end{definition}
\end{document} 