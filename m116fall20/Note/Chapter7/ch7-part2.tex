\documentclass[12pt]{article}
\usepackage{fullpage}
\usepackage{amsthm}
\usepackage{amsfonts,amsmath, amssymb,latexsym,mathrsfs}
\usepackage[margin=1.15in]{geometry}
\usepackage{enumitem}
\setlength{\parindent}{0pt}
\usepackage{tikz-cd}
\usepackage{fancyhdr}

\pagestyle{fancy}
\fancyhf{}
\rfoot{\thepage}
\rhead{Note for Calculus}



\theoremstyle{definition}
\newtheorem{thm}{Theorem}[section]

\newtheorem{prop}{Proposition}[section]


\theoremstyle{definition}
\newtheorem{definition}{Definition}[section]

\theoremstyle{remark}
\newtheorem*{remark}{Remark}

\theoremstyle{definition}
\newtheorem{example}{Example}[section]


\theoremstyle{definition}
\newtheorem{lem}{Lemma}[section]


\theoremstyle{definition}
\newtheorem{cor}{Corollary}[section]


\date{}
\title{Summary of 4.7 to 7.7}


\begin{document}
\maketitle

\section{L'Hopital's rule}

L’Hopital’s rule: If $f$ and $g$ are differentiable and (below $a$ can be $\pm \infty$)\\
i)$f(a) = g(a) = 0$ for finite $a$, \\
Or ii)$\lim_{x\to a} f(x)=\lim_{x\to a} g(x)= \pm \infty$,\\ 
Or iii)$\lim_{x\to \infty} f(x)= \lim_{x\to \infty} g(x) = 0$
then 
\[\lim_{x\to a}\frac{f(x)}{g(x)} = \lim_{x\to a} \frac{f'(x)}{g'(x)} \]

\subsection{Dominance}
We say that $g$ dominates $f$ as $x \to \infty$ if $\lim_{x\to \infty}f(x)/g(x) = 0$. 
\subsection{How to determine some bad limit?}

There are several types of the limits that is ''bad'' which requires L'Hopital's rule to calculate:
$0/0, \infty/\infty, \infty\cdot0$. Although the first two cases we can use L'Hopital's rule to calculate, the others we cannot use it directly.

Read the book, and there are several things that we can consider.
\begin{itemize}
	\item Consider taking log.
	\item Consider $1/f(x)$ so that we can transform $\infty$ to '$1/0$' or $0$ to '$1/\infty$'.
\end{itemize}

\section{Improper integral}
Formal definition of the improper integral I will let you read the book carefully, they are in the box. 
However, informally, there are two types of improper integral which we just interpret them as a limit.

\begin{itemize}
	\item The first case is where we have the limit of the integration goes to infinity, i.e. $\lim_{b \to \infty} \int^b_a f(x) dx$.
	\item The integrand goes to infinity as $x \to a$.
\end{itemize}

\subsection{Converges or diverges?}
The basic question that one want to know about the improper integral is basically is it well defined?

This turns to ask if an improper integral converges or not.

There are four ways people ususally use to check this fact.
\begin{enumerate}
	\item Check by definition, this means check the limit directly.
	\item $p$-test.\\
	\includegraphics*[width=0.9\textwidth]{1.png}
	\item Exponential decay test. 
	\[\int^\infty_0 e^{-ax} dx\] converges for $a>0$.
	\item Comparison test.\\
	If $f(x)\geq g(x) \geq 0$ on the interval $[a,\infty]$ then,\begin{itemize}
	\item If $\int^\infty_a f(x) dx$ converges then so does $\int^\infty_a g(x) dx$.
	\item If $\int^\infty_a g(x) dx$ diverges then so does $\int^\infty_a f(x) dx$.
	\end{itemize}
	\item Limit Comparison theorem.\\
	Limit Comparison Test. If $f(x)$ and $g(x)$ are both positive  on the interval $[a,b)$ where $b$ could be a real number or infinity.
	and
	\[\lim_{x\to b}\frac{f(x)}{g(x)} = C\] such that $0 < C < \infty$
	then the improper integrals $\int^b_a f(x) dx$ and $\int^b_a g(x) dx$ are either both convergent or both divergent.

\end{enumerate}


\end{document} 