\documentclass[12pt]{article}
\usepackage{fullpage}
\usepackage{amsthm}
\usepackage{amsfonts,amsmath, amssymb,latexsym,mathrsfs}
\usepackage[margin=1.15in]{geometry}
\usepackage{enumitem}
\setlength{\parindent}{0pt}
\usepackage{tikz-cd}
\usepackage{fancyhdr}




\theoremstyle{definition}
\newtheorem{thm}{Theorem}[section]

\newtheorem{prop}{Proposition}[section]


\theoremstyle{definition}
\newtheorem{definition}{Definition}[section]

\theoremstyle{remark}
\newtheorem*{remark}{Remark}

\theoremstyle{definition}
\newtheorem{example}{Example}[section]


\theoremstyle{definition}
\newtheorem{lem}{Lemma}[section]


\theoremstyle{definition}
\newtheorem{cor}{Corollary}[section]


\date{}
\title{Chapter 7: Integration}


\begin{document}
\maketitle




\section{Integration Techniques}

In this section, we will focus more on how to find the family of antiderivatives that is determined by the function we provided, since if we know such family, then given an initial value of $F(x)$, we can recover the specific antiderivative that we need.

\subsection{Guess and Check method}

Frankly speaking, this is gambling based on experience.\\
Although it is one of the fundamental method we use to solve problems, for example, it appears in also cryptography with the name Bruce force attack, and it also appears in computer science with the name Brute force search. However, as you probably know, this is usually not the most systematic or the most efficient way to the solution, and it usually requires either experience on the work or machine that do not rest. 

\subsection{Integration using Substitution}
Exercise as motivation: \textsection 6.4 35,37.

Similarly, the chain rule of derivative is also extremely useful in finding integral.

Theoretically, integration using substitution is using the philosophy that given an integration as $\int f(g(x)) g'(x) dx$, we will have $F(g(x))+C$ as its derivative, which is due to $\frac{d}{dx}(F(g(x))) = f(g(x))g'(x)$, and doing integration on both side will give us the ''substitution rule'' we want.

Technically, you want to recognize the ''$g(x)$'' part in your function, then substitude $g(x)$ as $w$, with the rule $dw=w'(x)dx =\frac{dw}{dx} dx$ (naively, we do have $\frac{dw}{dx} dx =dw$! However, this does not make much sense here which if you are particular interesting, you can ask me in person.)

Now let us consider an example, to see how is it actually applies.
Personally, my favorite almost trivial example is \[\int^{\infty}_0 e^{-x}dx=-\int_{0}^{-\infty}e^w -dw=\int^{0}_{-\infty}e^w dw=1.\text{ (Here I let }w=-x)\]

I will explain why is this one interesting in the next subsection, but here there is a warning you may see, \textbf{Check the up and low bound for your integration!} Let us see another example to see this clearer.

\[\int_{0}^{\pi/4}\frac{\tan^3 \theta}{\cos^2 \theta}d \theta = \int_{0}^{1}w^3dw=\frac{1}{4},\text{ ( Here I take } w = \cos \theta.)\]

Fast question: Can I do the following integrals? Why?$$(1) \int_{0}^{\pi/2}\frac{\tan^3 \theta}{\cos^2 \theta}d \theta  \text{ or } (2)\int_{-\pi/4}^{\pi/4}\frac{\tan^3 \theta}{\cos^2 \theta}d \theta$$


Suggested Problems: $\mathsection$7.1 some ex, 81,89,90-96,97,99,109,111,114,125,141,145,147

\subsection{Integration using partial fractions}

Small break after the introduction to substitution, we have this partial fractions.

Using the method of the substitution, we can already compute many weird integration like $\int \frac{2}{x-10} dx$.

Now, how about $\int \frac{1}{x^2-1} dx$? We are out of knowledge about how to do it right now, but there is an important method we ususally can apply, which is the partial fractions. 

The idea here is to break $\frac{1}{x^2-1}$ as a sum: $$\frac{1}{x^2-1}=\frac{1}{2(x-1) }+(-\frac{1}{2(x+1)})$$

But then now we can compute the integration in the question, since we can definitely compute $\int \frac{1}{2(x-1) } dx$ and $\int \frac{1}{2(x+1) }dx$.

If you run into repeated factors? The most general case (example): $$\frac{p(x)}{(x+c_1)^2(x+c_2)(x^2+c_3)}=\frac{A}{x+c_1}+\frac{B}{(x+c_1)^2}+\frac{C}{x+c_2}+ \frac{Dx+E}{x^2+c_3}$$


\subsection{Integration by parts}
If we call the integration using substitution the masterpiece established on the Chain Rule, then I have to say integration by parts is the same thing established on product rule.

Similar to the substitution, we reverse our thoughts here.

What is exactly happen in the integration by parts is that we are just undoing the product rules. Therefore, the key part is recognize your product rule carefully, 

Consider the product rule:
\[(uv)'=u'v+uv'=u'v+uv' \]
By some twisted, we will have the following formula for $uv'$ which we have not seen the advantages yet.
\[uv'=(uv)'−u'v\]
But now, integrate both sides will give us:
\[\int udv=\int(uv)′−\int vdu= uv - \int v du\], now if you remember what we talked about in the substitution chapter, you will immediately recognize that this is what we want.

Traditional Example:\[\int \ln(x) dx\] \[\int \cos^2 x\]

Suggested Problems: $\mathsection$ 7.2 49,53,55,65,69,73

\newpage

\section{Integration Approximation}

Estimate the integral is based on the fact that after taking limits of Left/Right/Mid/Trapezoidal Riemann Sum, they are gonna equals and will give the actual integration value, thus if we just compute the Riemann sums in finite many block, we will have a value that is pretty much close to the actual integration, and we can refine our calculation by taking more blocks.

Reminder: we have then four ways of estimating an integral using a Riemann Sum:
\begin{enumerate}
\item LEFT(n)
\item RIGHT(n)
\item MID(n)
\item TRAP(n)
\end{enumerate}
Think about what they are?
\textbf{Think about Pictures!}

Is your estimation over/under?\\

\textbf{Again, Pictures!}(It probably will never make sense without picture.)
\begin{enumerate}
\item If the graph of $f$ is increasing on $[a,b]$, then $$LEFT(n)\leq \int^b_a f(x) dx \leq RIGHT(n)$$
\item If the graph of $f$ is decreasing on $[a,b]$, then $$RIGHT(n)\leq \int^b_a f(x) dx \leq LEFT(n)$$
\item If the graph of $f$ is concave up on $[a,b]$, then $$MID(n)\leq \int^b_a f(x) dx \leq TRAP(n)$$
\item If the graph of $f$ is concave down on $[a,b]$, then $$TRAP(n)\leq \int^b_a f(x) dx \leq MID(n)$$

\end{enumerate}

Suggested Problems: $\mathsection$7.5	1,3,5,16,17,25



\end{document} 