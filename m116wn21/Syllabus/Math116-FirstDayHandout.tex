%%This is a standard LaTeX2e article document template. personal version 12/5/200%%
\documentclass[11pt,twoside]{article}
%%%%%%%%%%%%%%%%%%%%%%%%%%%%%%%packages%%%%%%%%%%%%%%%%%%%%%%%%%%%%%%%%%%%%%%%%%%%%%%%%%%%%%%%%%%
\usepackage[top=0.5in,bottom=0.6in,left=1in,right=1in]{geometry}

\pagestyle{empty}

\usepackage{latexsym}
\usepackage{amssymb}
\usepackage{amsfonts}
\usepackage{amstext}
\usepackage{multicol}
\usepackage{hyperref}
%%%%%%%%%%%%%%%%%%%%%%%%%%%%%%%formatting%%%%%%%%%%%%%%%%%%%%%%%%%%%%%%%%%%%%%%%%%%%%%%%%%%%%%%%
\setlength{\topmargin}{-.5in}        %%%  This sets all the spacing stuff to use the page more
\setlength{\oddsidemargin}{0in}    %%%  efficiently than the normal "article" setup would.
\setlength{\evensidemargin}{0in}   %%%  It's OK to play with these some.
\setlength{\textheight}{10in}     %%%
\setlength{\textwidth}{6.75in}     %%%
\setlength{\headsep}{0in}          %%%
\setlength{\headheight}{0in}       %%%
%\setlength{\footskip}{0in}         %%%s
\newcommand\topic[1]{\noindent{\bf #1}}
%%%%%%%%%%%%%%%%%%%%%%%%%%%%%%%%%%%%%%%%%%%%%%%%%%%%%%%%%%%%%%%%%%%%%%%%%%%%%%%%%%%%%%%%%%%%%%%
\title{Math 116 W21 First Day}

\begin{document}
\hspace{1ex}

\centerline{\bf \large Math 116 -- Calculus II  -- Section 042 -- Winter 2021
}
\vspace{3pt}

\hrule
\vspace{5pt}


\begin{tabular}{p{2in}p{2in}p{2.1in}}
{\bf Instructor}: {Yiwang Chen}  &{\bf Email}: \href{mailto:email}{yiwchen@umich.edu}\\
 \multicolumn{3}{l}{ {\bf Class:} 
 11:30-13:00 MTTh} \\  
 \multicolumn{3}{l}{ {\bf Office Hours:} 
 TBD} \\
\end{tabular}

\vspace{7pt}
\hrule
\vspace{7pt}

\topic{Course Description:} The course 
presents concepts of calculus from four points of view: geometric (graphs), numeric (tables), symbolic (formulas), and verbal descriptions. Students will develop their reading, writing and questioning skills, as well as their ability to work cooperatively. Topics include techniques of integration, application of these, sequences and series, Taylor series, polar coordinates, and parametric coordinates.
\vspace{5pt}

\topic{Text:} {\it Calculus: Single Variable} by Hughes-Hallett, Gleason, et al., \underline{7th} Edition, published by John Wiley and Sons.  ISBN:  978-0470-88864-3\\  
You \emph{must} keep up with the reading assignments.  Please be ready to access your text in class every day.

\vspace{5pt}
\topic{Calculator:} You may use \url{https://www.desmos.com/calculator} or equivalent software.  A graphing calculator may also be used, but you are responsible for learning how to operate it.

\vspace{5pt}

\topic{Course Website:}  \url{http://www.math.lsa.umich.edu/courses/116/}\\
This site has important information including key dates for the course, important links, and homework assignments. Please visit it soon and read the information there! It will be updated with
other information as the semester progresses.

\vspace{5pt}

\noindent Our section has a Canvas page at \url{http://canvas.umich.edu}. I will post information for our section on that page.  Videos will be posted there for you to watch before coming to each class.  % You will probably want to add any other things that you plan to use Canvas for.


\vspace{5pt}

\topic{Email:} You are responsible for information contained in course email messages sent to your official ``umich.edu'' email address.  Please check your email regularly.

\vspace{5pt}
\topic{\bf Grading Policy:}  All sections of Math 116 use the same grading guidelines to standardize the evaluation process. %A complete explanation of the grading policy is given in the Student Guide on the course website.  Look under the heading ``Grading System''.
We will While 95\% of your grade will be determined uniformly across sections, 5\% of your grade will be determined by your Section Component. More details of how your section component will be calculated are outlined below. 

\vspace{5pt}

\topic{Homework:}   The only way to really LEARN mathematics is to DO mathematics.
  \vspace{-2ex}

 \begin{itemize}
 \item Along with the math you will do during class, 
 \underline{web homework} (WeBWorK) will be assigned from each section that we cover.  These assignments are accessed and completed online.  (There is a link on the Course Website.)  In order to do well in this class, you {\bf must} keep up with these assignments.   The assignments will usually be open for about a week, and the deadlines are firm. 
 
Reading homework assignments will be due on the day material is to be covered in class. Along with the videos on Canvas, reading the textbook and completing the reading homework are essential parts of preparing for class each day. 

The Reading homework assignments will be together worth 5\% of your overall grade. Web homework assignments will count for 10\% of your grade.   %You may decide whether or not to include the early assignments as a quiz grade
  \vspace{-1ex}

%\end{itemize}

%\noindent 
\item In addition, you will be given regular \underline{team homework} assignments, and a large portion (60\%) of the Section Component of your grade will be based on these assignments. We will discuss team homework arrangements, policies, and expectations further in a later class. %You may decide for yourself how quizzes, team homeworks, etc. contribute to your section component (together worth 5\% of the overall grade).
\end{itemize}

\vspace{-2pt}
\topic{Quizzes:}  There will be a quiz in class almost every week.  No make-ups will be given, but I will drop your {\bf two lowest} quiz scores when I compute your final grade.  Quizzes (together with any other graded assignments not already described above) will count for 30\% of the Section Component of your grade.\\ %Decide for yourself how much quizzes are worth, and how many scores to drop, if any.

\vspace{-2pt}
\topic{Participation:} An important part of this class is learning how to communicate mathematics and so we will spend a lot of time working in groups. Students who regularly miss class or don't actively participate in class struggle to keep up with the material and tend to perform poorly on exams. For these reasons, in-class participation will count for 10\% of the Section Component of your grade. If you have to miss class for whatever reason, please email me \emph{before the start of the class} and you may be excused. There will also be allowances made for internet troubles, but please let me know as soon as possible when you are experiencing these issues. \\ %You may decide for yourself whether or not participation should count towards the grade. 

 \topic{Uniform Exams:} 
 \vspace{2pt}
 
\hspace{-5ex}
\begin{tabular}{lll}
{First Exam (20\%  of grade):} & Monday, February 22 & 8:00 - 9:30 pm \\
{Second Exam (25\% of grade):} &  Monday, March 29 & 8:00 - 9:30 pm  \\
{Final Exam (30\% of grade):} & Friday, April 23 & 8:00 - 10:00 am 
\end{tabular}

\vspace{0.1in} 

%The exams are 90 minutes long but students will be given an extra 30 minutes to scan and upload solutions. Please encourage students to make sure they are available for the full 2 hours. There will be an alternate time exams for students in different time zones but thise are still to be scheduled.

\noindent The dates for the exams are absolutely firm.  Make plans {\sc now} to be certain these dates are in your calendar.  Generally, only students with a regularly scheduled
class are accommodated at an alternate time. Anyone with a
regularly scheduled class during these times should let me know immediately. Note that travel is \emph{not} a sufficient excuse to have an exam scheduled on a different day.
Missing an exam with an unapproved or undocumented excuse {\em will result in a grade penalty} in the course. %May need to include info about time zones here

\vspace{7pt}
\topic{Gateway Exams:}  There are two gateway tests for Math 116 students. The first reviews differentiation and the second is on techniques of integration. The tests will be online, and you will be required to submit your written work for a passing attempt. You will be able to attempt each test multiple times, but each will be open for a short period of time (around two weeks). Students will lose a third of a letter grade (on their grade for the course) for failing to pass the proctored first gateway during its open period. Successful completion of the second Gateway will count as 5\% of your grade. I will give more details about these tests in class.


\vspace{7pt}

\topic{Academic Integrity:} According to the \emph{LSA Community Standards of Academic Integrity}, the College ``prohibits all forms of academic dishonesty and misconduct.  Academic dishonesty may be understood as any action or attempted action that may result in creating an unfair academic advantage for oneself or an unfair academic advantage or disadvantage for any other member or members of the academic community.''  Do {\sc not} cheat.  If you cheat in this class, you risk failing the course. If you have any questions about what is, or is not, allowed in this course, please ask.

For this online semester in particular, please note carefully that the following are all considered to be forms of cheating:

\hspace{-1ex}$\bullet$ Any communication with any other person during quizzes or exams;

\hspace{-1ex}$\bullet$ Asking for help with exam or quiz problems on any online question and answer forum;

\hspace{-1ex}$\bullet$ Using any unauthorized online resources.


\vspace{7pt}

\topic{In Class:}  You are required to come to class.  Students who do not attend class rarely succeed in this course.  If you must miss class for any reason, it is your responsibility to get notes from a classmate and catch up on the material.  You are also responsible for the information given in any announcements made during class.  
A few points to remember:

\hspace{-1ex}$\bullet$ Come prepared to work and participate (This includes doing the assigned reading and homework before arriving to class).

\hspace{-1ex}$\bullet$ Participate in all class activities, including in-class group work, presentations, and discussion.

\hspace{-1ex}$\bullet$ Come to class on time and do not leave early.  
If for some reason you absolutely have to arrive 

\hspace{1ex}late or leave early, please let me know in advance and do so discretely.

\hspace{-1ex}$\bullet$ Treat your peers and instructor with respect and abide by all course policies.

\vspace{7pt}

\topic{Math Lab:}  Free tutoring from the Mathematics
Department. See \url{https://lsa.umich.edu/math/undergraduates/course-resources/math-lab.html} for more information.

\begin{tabular}{lll}
Hours: & Monday through Thursday & 11:00 am -- 4:00 pm and 7:00 pm -- 10:00 pm \\ 
       &         Friday 			& 11:00 am -- 4:00 pm\\
       &         Sunday 			& 7:00 pm -- 10:00 pm \\ 
\end{tabular}

\vspace{7pt}

%You may want to include one of the two following statements:

%\topic{Lecture Recordings}: Course lectures may be audio/video recorded and made available to other students in this course. As part of your participation in this course, you may be recorded. If you do not wish to be recorded, please contact [instructor/gsi email address] the first week of class (or as soon as you enroll in the course, whichever is latest) to discuss alternative arrangements.
%\vspace{7pt}

%\topic{Lecture Recordings}: Students are prohibited from recording/distributing any class activity without written permission from the instructor, except as necessary as part of approved accommodations for students with disabilities. Any approved recordings may only be used for the student�s own private use.
%\vspace{7pt}

\topic{Accommodations for Students with Disabilities:}  
If you think you need an accommodation for a disability, please let me know as soon as possible. In particular, a Verified Individualized Services and Accommodations (VISA) form must be provided to me \underline{at least two weeks} prior to the need for a test/quiz accommodation. The Services for Students with Disabilities (SSD) Office (G664 Haven Hall; http://ssd.umich.edu/) issues VISA forms.
\vspace{7pt}

%The following is not required to appear in your handout but you should consider including it anyway 
\topic{Mental Health and Wellbeing} The University of Michigan is committed to advancing the mental health and wellbeing of its students. If you or someone you know is feeling over-whelmed, depressed, and/or in need of support, services are available. For help, contact Counseling and Psychological Services (CAPS) at (734) 764-8312 and \url{https://caps.umich.edu/}. You may also consult University Health Service (UHS) at (734) 764-8320 and \url{https://www.uhs.umich.edu/mentalhealthsvcs}, or for alcohol or drug concerns, see \url{www.uhs.umich.edu/aodresources}. For a more comprehensive listing of the broad range of mental health services
available on campus, please visit: \url{http://umich.edu/~mhealth/}.

\vspace{7pt}

\topic{Final Note:} These are challenging times, and I understand that you may have potential concerns, or have unforeseen issues this semester. I will work hard to accommodate your needs as best I can and I aim to make the semester as interesting, fun and rewarding as it can be. I look forward to working with and getting to know you all.   If you have any questions or concerns during the term, please do not hesitate to contact me!  

\end{document}